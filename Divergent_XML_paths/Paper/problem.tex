%%-*-latex-*-

\section{Context and problem}
\label{problem} 

\subsection{Streaming of \XML elements} 

A \emph{tag} is a pair made of a \emph{kind}, noted \(\kappa\), and a
status ``opening'' or ``closing''. A \emph{element}, noted
\(\varepsilon\), is a triple made of an opening tag, some contents and
a closing tag of the same kind as the opening tag. The contents is
made of a series of plain text and other (embedded) elements. Elements
can be empty, in which case the contents is empty. By extension, an
element is said to be of kind \(\kappa\) if its tags are of kind
\(\kappa\). Let us define the function \(K\) on elements which gives
the kind of its argument. A set of kinds is noted \(\cal K\). Kinds
are totally ordered. For example, let us consider the following
original \XML file where there is only one tag per line and lines are
numbered. The right column shows the first step in the streaming of
this file.  
{\small\begin{verbatim}
 1 <c> text1                          <c range="1--22"> text1 </c>
 2   <a> text2                        <a range="2--13"> text2 </a>
 3     <b> text3                      <b range="3--6"> text3 </b>
 4       <a> text4                    <a range="4--5"> text4 </a>
 5       </a>              
 6     </b>
 7     <b> text5                      <b range="7--8"> text5 </b>
 8     </b>
 9     <c> text6                      <c range="9--12"> text6 </c>
10       <a> text7                    <a range="10--11"> text7 </a>
11       </a>
12     </c>
13   </a>
14   <a> text8                        <a range="14--15"> text8 </a>
15   </a>
<16   <b> text9                        <b range="16--21"> text9 </b>
17     <c> text10                     <c range="17--20"> text10 </c>
18       <b> text11                   <b range="18--19"> text11 </b>
19       </b>
20     </c>
21   </b>
22 </c>
\end{verbatim}}
\noindent Here, the kinds of tag are \(a\), \(b\) and \(c\), i.e.,
 \({\cal K} = \{a, b, c\}\). Elements are streamed by increasing order
 of their opening tag line in the \XML file. In order not to lose
 information, let us record the line numbers of the opening tag and
 the matching closing tag into an attribute \texttt{range}. This
 scheme works if we assume one tag per line, but, in general, we do
 not have a single tag per line, so we cannot take as unique position
 the line number. In this case, tags must be numbered by order of
 appearance in the file. Let us assume that there is a function \(T\)
 which takes an element and returns its textual contents (a series of
 pieces of text). This allows to abstract away the text in the
 element.

Let \(L(\varepsilon)\) and \(U(\varepsilon)\) be respectively the
 lower bound of the range of element \(\varepsilon\) and the upper
 bound of the range of \(\varepsilon\). By definition, element
 \(\varepsilon_1\) is said to be \emph{lower} than element
 \(\varepsilon_2\), noted \(\varepsilon_1 < \varepsilon_2\), if and
 only if \(L(\varepsilon_1) < L(\varepsilon_2)\). Elements are
 streamed by increasing lower bounds of their range.

Let us note \(a_n\) the \(n\)-th element of kind \(a\), \(b_n\) then
 \(n\)-th element of kind \(b\) and \(c_n\) the \(n\)-th element of
 kind \(c\), as it is usual in the multiple\hyp{}streams
 view. (Al\hyp{}Khalifa et al. \cite{AlKhalifa:2002} assume that the
 database provides multiple streams of such sorted elements of the
 same kind of tag.) With this approach we say that we have here three
 streams of different kinds: [\(a_1\), \(a_2\), \(\dots\)], [\(b_1\),
 \(b_2\), \(\dots\)] and [\(c_1\), \(c_2\), \(\dots\)], where \(a_1\),
 \(b_1\) and \(c_1\) are the first elements of their stream. By
 interleaving the streams into one stream so that the elements are
 totally ordered, the previous streams become the unique series
 [\(c_1\), \(a_1\), \(b_1\), \(a_2\), \(b_4\), \(c_2\), \(a_3\),
 \(a_4\), \(b_2\), \(c_3\), \(b_3\)]. (We can easily achieve such a
 stream from multiple streams by picking repeatedly the lowest element
 among the first elements of the streams.) In this paper, we prefer to
 assume a unique stream of sorted elements. This is more precise,
 since it specifies the total ordering of the elements, not just
 between those of same kind. Nevertheless, for the sake of clarity, we
 keep the subscripting of tags as in the multiple\hyp{}streams
 framework. The following left table shows the stream as the
 interleaving of multiple streams, whilst the right table displays the
 same stream with a single\hyp{}stream view, i.e., considering a
 generic series of elements [\(\varepsilon_1\), \(\varepsilon_2\),
 \(\dots\), \(\varepsilon_{11}\)], with a supplementary column \(K\)
 for the kind of element.
\[
\begin{array}{|c||c|c|c|c|c||c|c|c|c|}
\hhline{~---~~----}
\multicolumn{1}{c||}{} & T & L & U &
\multicolumn{1}{c}{} & \multicolumn{1}{c||}{} & K & T & L & U\\
\hhline{=::===~=::====}
c_1 & \texttt{text1}  & 1 & 22 & & \varepsilon_1 & c & \texttt{text1} & 1 & 22\\
\hhline{-||---~-||----}
a_1 & \texttt{text2}  & 2 & 13 & & \varepsilon_2 & a & \texttt{text2} & 2 & 13\\
\hhline{-||---~-||----}
b_1 & \texttt{text3}  & 3 & 6 & & \varepsilon_3 & b & \texttt{text3} & 3 & 6\\
\hhline{-||---~-||----}
a_2 & \texttt{text4}  & 4 & 5 & & \varepsilon_4 & a & \texttt{text4} & 4 & 5\\
\hhline{-||---~-||----}
b_4 & \texttt{text5}  & 7 & 8 & & \varepsilon_5 & b & \texttt{text5} & 7 & 8\\
\hhline{-||---~-||----}
c_2 & \texttt{text6}  & 9 & 12 & & \varepsilon_6 & c & \texttt{text6} & 9 & 12\\
\hhline{-||---~-||----}
a_3 & \texttt{text7}  & 10 & 11 & & \varepsilon_7 & a & \texttt{text7} & 10 & 11\\
\hhline{-||---~-||----}
a_4 & \texttt{text8}  & 14 & 15 & & \varepsilon_8 & a & \texttt{text8} & 14 & 15\\
\hhline{-||---~-||----}
b_2 & \texttt{text9}  & 16 & 21 & & \varepsilon_9 & b & \texttt{text9} & 16 & 21\\
\hhline{-||---~-||----}
c_3 & \texttt{text10} & 17 & 20 & & \varepsilon_{10} & c & \texttt{text10} & 17
& 20\\
\hhline{-||---~-||----}
b_3 & \texttt{text11} & 18 & 19 & & \varepsilon_{11} & b & \texttt{text11} & 18
& 19\\
\hhline{-||---~-||----}
\end{array}
\]


\subsection{Containment and disjointedness} 

Since the stream of elements is computed from a valid \XML file, any
 two elements in the stream are either in a \emph{containment} or else
 a \emph{disjointedness} (non\hyp{}overlapping) relationship. By
 definition, an element \(\varepsilon_1\) is contained in
 \(\varepsilon_2\) (or ``is a descendant of''), noted \(\varepsilon_1
 \sqsubset \varepsilon_2\), if \(L(\varepsilon_2) < L(\varepsilon_1)\)
 and \(U(\varepsilon_1) < U(\varepsilon_2)\). Containment is neither
 reflexive (it is strict inclusion) nor symmetric, but it is
 transitive. Elements \(\varepsilon_1\) and \(\varepsilon_2\) are
 disjoint, noted \(\varepsilon_1 \, \sharp \, \varepsilon_2\), if and
 only if \(U(\varepsilon_1) < L(\varepsilon_2)\) or \(U(\varepsilon_2)
 < L(\varepsilon_1)\), i.e., \(\varepsilon_1 \not\sqsubset
 \varepsilon_2\) and \(\varepsilon_2 \not\sqsubset
 \varepsilon_1\). Disjointedness is neither reflexive nor transitive,
 but it is symmetric.


\subsection{Problem statement}

Let us note \(\cal E\) the infinite set of all possible \XML elements.
\begin{wrapfigure}[7]{r}{0pt}
\fbox{\includegraphics[bb=71 693 149 721]{pattern}}
\caption{Query \(\query{\kappa_1}{\kappa_2, \dots, \kappa_n}\)}
\label{pattern}
\end{wrapfigure}
 A \emph{pattern}, or \emph{query}, graphically represented in
 figure~\ref{pattern}, is a tuple of \XML kinds of tag \(\kappa_i\),
 noted \(\query{\kappa_1}{\kappa_2, \dots, \kappa_n}\). Let \(\cal A\)
 be the function that returns the arity of a query, i.e., the number
 of kinds it contains. Let us note \(\Inter{m}{n}\) the integer
 interval from \(m\) to \(n\). A stream of \XML elements is a series
 [\(\varepsilon_1\), \(\varepsilon_2\), \(\dots\)], such that
 \(\forall i,j \in \mathbb{N}.(0 < i < j \Rightarrow \varepsilon_i <
 \varepsilon_j)\). A \emph{complete match} of a query \(q\), noted
 \(\mu_q\), is a finite mapping from the indexes of the kinds in \(q =
 \query{\kappa_1}{\kappa_2, \dots, \kappa_{{\cal A}(q)}}\) to \(\cal
 E\), i.e., \(\mu_q: \Inter{1}{{\cal A}(q)} \rightarrow {\cal E}\),
 such that
\begin{align}
\forall i \in \Inter{1}{{\cal A}(q)}.& K \circ \mu_q(i) = \kappa_i
\label{match:tags}\\
\forall i \in \Inter{2}{{\cal A}(q)}.& \mu_q(i) \sqsubset \mu_q(1)
\label{match:desc}\\
\forall i,j \in \Inter{2}{{\cal A}(q)}.& \mu_q(i) \not\sqsubset \mu_q(j)
\label{match:div}
\end{align}
For example, let \(q = \query{a}{b,b}\) and the input stream [\(d_1\),
 \(a_1\), \(c_1\), \(b_1\), \(c_2\), \(a_2\), \(b_2\)], then there is
 a match \(\mu_q\) which satisfies \(\mu_q(1) = a_1\), \(\mu_q(2) =
 b_1\), \(\mu_q(3) = b_2\). Alternatively, we can write instead
 \(\mu_q = [1 \mapsto a_1, 2 \mapsto b_1, 3 \mapsto b_2]\) (the order
 of the bindings in the map is not meaningful). This is an
 \emph{ordered match} because it enjoys the additional condition:
 \(\forall i,j.(i < j \Rightarrow \mu_q(i) < \mu_q(j))\). Otherwise, a
 match is said to be \emph{unordered}. For example, \([1 \mapsto a_1,
 2 \mapsto b_2, 3 \mapsto b_1]\) is an unordered
 match. Conditions~\eqref{match:tags} and~\eqref{match:desc} are the
 standard interpretation of \XPath queries. We add here
 condition~\eqref{match:div}, which constraints the elements matching
 the pattern descendants to be pairwise disjoint. Our algorithm
 takes as input a query and a stream. It reads the elements from the
 stream one by one and it outputs as soon as possible all the matches
 from the pattern to the stream read so far. If no match is found
 after reading an element, the algorithm can be run again on the
 remaining stream with a data structure that keeps the intermediary
 results. This way, it is an on\hyp{}line algorithm. This problem is
 perhaps more intuitively presented as a property of an \XML tree.

\subsection{The \XML trees}

The tree representation of an \XML document is built considering that
 an element maps to a tree whose root is labeled with a tag and that
 the directly contained elements map to subtrees ordered according to
 the order of elements defined above, i.e., siblings are increasingly
 ordered from left to right. Figure~\ref{xml_tree} is a tree made from
 the stream [\(d_1\), \(a_1\), \(c_1\), \(b_1\), \(c_2\), \(a_2\),
 \(b_2\), \(d_2\), \(a_3\), \(b_3\), \(c_3\)]. The order of the
 elements always corresponds to a \emph{preorder traversal} of the
 original tree. Also, two nodes are in a containment relationship if
 and only if they both belong to a rooted path, i.e., a path including
 the root. Assume the tree in figure~\ref{xml_tree}, and a query
 \(\query{a}{b,c}\), then the unique ordered match is \([1 \mapsto
 a_1, 2 \mapsto b_1, 3 \mapsto c_2]\).
\begin{wrapfigure}[11]{r}{0pt}
\centering
\fbox{\includegraphics[bb=71 651 141 721]{xml_tree}}
\caption{\XML tree with tags \(a\), \(b\), \(c\) and \(d\).}
\label{xml_tree}
\end{wrapfigure}
\noindent We can think of a match as a map from the nodes of the query
 tree to some nodes of the \XML tree (condition~\eqref{match:tags})
 such that the descendants in the query tree are mapped to descendants
 of the mapping of the query root (condition~\eqref{match:desc}) and
 such that these descendants are located in paths diverging from the
 mapping of the query root (condition~\eqref{match:div}). Coming back
 to our example, if we are interested in the unordered matches as
 well, we must add \([1 \mapsto a_1, 2 \mapsto b_2, 3 \mapsto
 c_1]\). By contrast, if we interpret the query as in \XPath, i.e.,
 allowing containment between descendants, we would have to add \([1
 \mapsto a_1, 2 \mapsto b_1, 3 \mapsto c_1]\), \([1 \mapsto a_1, 2
 \mapsto b_2, 3 \mapsto c_2]\) and \([1 \mapsto a_3, 2 \mapsto b_3, 3
 \mapsto c_3]\).

\subsection{Rightmost branches}

The \emph{rightmost branch} of a tree is the longest rooted path made
 of the successive rightmost children. In the example at
 figure~\ref{xml_tree}, the rightmost branch is [\(d_1\), \(a_3\),
 \(b_3\), \(c_3\)]. Adding a new node, i.e., the next available
 element
\begin{wrapfigure}[11]{r}{0pt}
\centering
\subfloat[Initial]{
  \fbox{\includegraphics[bb=71 651 104 721]{rightmost_branch}}
  \label{initial_branch}}
\quad
\subfloat[Extension]{
  \fbox{\includegraphics[bb=71 651 118 721]{rightmost_extension}}
  \label{extension}}
\quad
\subfloat[Expansion]{
  \fbox{\includegraphics[bb=71 651 122 721]{rightmost_expansion}}
  \label{expansion}}
\caption{Rightmost branches (in bold).\label{rightmost_branches}}
\end{wrapfigure}
in the input stream, changes only the rightmost branch because the
elements are ordered by increasing positions of their opening tags
(preorder traversal). Consider previous stages of our \XML tree in
figure~\ref{rightmost_branches}: before adding \(a_2\)
(sub\hyp{}figure \ref{initial_branch}), after adding \(a_2\)
(sub\hyp{}figure~\ref{extension}) and after adding \(b_2\)
(sub\hyp{}figure~\ref{expansion}). This sequence illustrates the fact
that inserting a node either extends the rightmost branch (as adding
\(a_2\)) or creates a new branch rooted in the previous rightmost
branch (as adding \(b_2\)). We call the first case a \emph{rightmost
  extension} and the second is known as \emph{rightmost expansion} of
a tree in the mining literature (see Asai et al. \cite{Asai:2003}).
\begin{figure}
{\footnotesize
\begin{verbatim}
let $doc := fn:doc("file:///tmp/doc.xml")
for $a in $doc//a,
    $b in $a//b,
    $c in $a//c except $a//$b//c
                except $a//c[descendant::b=$b
                             and descendant::b/@order=$b/@order]
return <disjoint_elements>
       {(element {fn:node-name($a)}{$a/@order},
         element {fn:node-name($b)}{$b/@order},
         element {fn:node-name($c)}{$c/@order})}
       </disjoint_elements>
\end{verbatim}
}
\caption{\XQuery encoding of \(\query{a}{b, c}\)}
\label{problem:xquery}
\end{figure}


\subsection{XQuery}

The kind of query we propose to answer in this article cannot be
 expressed by the standard semantics of \XPath, but it can be
 expressed in \XQuery. For example, consider the query \(\query{a}{b,
 c}\) in figure~\ref{binary_query}. It can be matched against an \XML
 file named \texttt{\small /tmp/doc.xml} corresponding to the \XML
 tree of figure~\ref{xml_tree} in the following manner, assuming that
 the attribute \texttt{\small order} records the subscripts, e.g.,
 \texttt{\small <a order="3"> ... </a>} stands for \(a_3\). See
figure~\ref{problem:xquery}.
