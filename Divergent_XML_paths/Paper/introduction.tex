%%-*-latex-*-

\section{Introduction}

\noindent The \XPath 2.0 expression \texttt{\small a//b|a//c} is
interpreted as ``\texttt{\small a//b} and \texttt{\small a//c}'', if
we omit the unique target semantics (\XPath expressions evaluate to a
a sequence of nodes with
\begin{wrapfigure}[6]{l}{45pt}
\centering
\includegraphics[bb=71 694 100 721]{binary_query}
\caption{\textsf{\small a//b[//c]}\label{binary_query}}
\end{wrapfigure}
same tag, here \texttt{\small b}). Many research
papers~\cite{Cooper:2001,Li:2001,Zhang:2001} are devoted to match
efficiently these expressions, considered as tree patterns or queries,
against \XML trees as a whole or streams of \XML elements, either by
solving the binary joins and then stitching them back to answer the
original query~\cite{AlKhalifa:2002,Chien:2002,Jiang:1:2003,Wu:2003}
or by considering them as a
whole~\cite{Bruno:2002,Jiang:2:2003,Jiao:2005}.  The standard
interpretation of the \XPath expressions implies that in the previous
example, \texttt{\small b} and \texttt{\small c} can be in an
ancestor/descendant relationship, and the published algorithms indeed
stick to this interpretation. But it is striking to realise that all
the authors, to our knowledge, assume in their examples that the nodes
matched by \texttt{\small b} and \texttt{\small c} are \emph{not} in
an ancestor or descendant relationship. This let us surmise that this
tacit assumption in the presentation (not present in the algorithm) is
often closer to what the query specifier or reader has in mind than
the usual interpretation. In other words, the reader of an \XPath
query probably tends to represent in his mind the query as a tree, as
in the figure~\ref{binary_query}, and to assume that the matching
subtree is isomorphic to it, instead of reading the query as a set of
binary joins that, once stitched together, can lead to a degenerate
case, like a path. In particular, the expression \texttt{\small
  a//b[//b]} is equivalent to \texttt{\small a//b} and there is no way
in \XPath to actually denote two different \texttt{\small b} nodes on
two divergent paths, i.e., no path including the other, as the graphic
representation seems to suggest. But can we rely solely on this
psychological assumption to support another interpretation of the
\XPath expressions? We believe that, beyond the useful purpose of
avoiding misunderstandings by making more assumptions explicit, it is
indeed useful to assume sometimes that the matched subtree is
isomorphic to the query tree because this allows some semantics to be
included in the query. As an illustration, consider the \MathML trees
corresponding to the expressions \(\cos(x) + \sin(y)\) and
\(\cos(\sin(x)) + y\) shown in figure~\ref{mathml}. These trees,
despite having a structure close enough to be matched by the same
pattern, have very different mathematical meanings. By interpreting
the query pattern in a more restrictive way than in \XPath, we can get
less matches. This amounts to push some semantics into the syntax. For
example, if we want to match the difference between a possibly nested
sine application and a possibly nested cosine application, \emph{but
  not composing them}, we can say that the two nodes \textsf{apply} in
the query cannot match nodes which are in a descendant or ancestor
relationship, and then the query only matches the tree in
sub-figure~\ref{cos_minus_sin}. Furthermore, since we know that the
subtraction is not commutative when interpreting these formulas, we
could impose further that the two matched nodes \textsf{apply} must be
in the \emph{same order} as in the pattern, that is, the relative
order of these subtrees is the document order, which leads here to no
match at all.
\begin{figure}[b]
\centering
\subfloat[\(\cos(x) - \sin(y)\)\label{cos_minus_sin}]
         {\includegraphics[bb=71 632 168 721]{cos_minus_sin}}
\qquad\qquad
\subfloat[\(\cos(\sin(x)) - y\)\label{cos_sin_minus}]
         {\includegraphics[bb=71 632 154 721]{cos_sin_minus}}
\qquad
\subfloat[Pattern with variables \(a\) and \(b\).]
         {\includegraphics[bb=61 632 180 721]{mathml_query}}
\caption{Two \MathML trees and a query tree}
\label{mathml}
\end{figure}
\noindent We can think of some application domains that could benefit
from this new interpretation, but real cases are hard to find because
(\emph{i}) the users, often companies, are reluctant to share their
databases, (\emph{ii}) too many \XML users still consider \XML mainly
as a common format for information interchange and not as a
semi-structured data that can be queried, hence it is far too common
to see very flat \XML data models even if the original information is
highly structured (like a molecule)\footnote{Perhaps this situation is
  also the result of an automated migration from relational
  databases.}.

The paper is structured as follows. We start with the next
section~\ref{problem} by stating mode precisely the problem we want to
solve; then we propose a naive solution in section~\ref{naive};
follows a stack\hyp{}based algorithm, in section~\ref{stack}, and an
index\hyp{}based procedure, in section~\ref{table}; this paper being
concluded in section~\ref{conclusion} with a short comparison of the
proposed algorithms and plans for future work.
