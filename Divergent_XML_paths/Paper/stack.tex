%%-*-latex-*-

\section{A stack-based algorithm}
\label{stack}


\subsection{The rightmost branch as a stack of elements}

Since only the rightmost branch changes throughout insertions, we
can keep in memory only the rightmost branch, instead of the whole
\XML tree. The way it changes also allows us to implement it as a
\emph{stack}. Al-Khalifa et al. \cite{AlKhalifa:2002} used the same
observation. The bottom of the stack is the root element and the top
is the last inputted element. If the newly inputted element is
contained in the rightmost leaf (i.e. it extends the rightmost
branch), it is pushed onto the stack, otherwise elements are popped
until the top element contains the new element, which is then pushed
(this realises the rightmost expansion and saves some memory).


\subsection{Attributes of stack elements}

Let us associate some \emph{partial matches} to the elements in the
stack (i.e. the rightmost branch of the \XML tree), i.e. matches with
some descendants still unmatched. These partial matches are of two
kinds. First, the \emph{local attributes} are partial matches made by
trying to complete the previous partial matches containing no element
of the stack with the current element. Second, the \emph{synthesised
  attributes} are the partial matches involving the contained elements
as well as the current element if it is a possible match for the
pattern root. In order to illustrate the algorithm, we shall assume a
query \(\query{a}{b,c}\). A complete match \([1 \mapsto a_1, 2 \mapsto
  b_1, 3 \mapsto c_2]\) appears as \(\mathbf{a_1b_1c_2}\) for the sake
of brevity. Element \(d_1\) has no attribute since its kind is not in
the query. Element \(a_1\) has no local attributes and it has some
synthesised attributes which are partial matches made of itself and
its contained elements which are not in the stack, i.e. \(c_1\) and
\(b_1\). Element \(c_2\) has some local attributes made by combining
itself with the synthesised attributes of its ancestors in the stack,
i.e. \(a_1\) and \(d_1\), and it has no synthesised attributes itself
since it has no contained elements in the stack (yet).


\subsection{Stack specification}

Formally, let \proc{Empty} denote any empty stack; \(\proc{Push}
(\id{x}, \id{S})\) the stack whose top element is \id{x} and remaining
stack is \id{S}; \(\proc{Pop} (\id{S})\) the pair whose first
component is the top element of stack \id{S} and the second is the
remaining stack\footnote{This definition amounts to say that
  \proc{Pop} is the exact inverse function of \proc{Push}, that is to
  say \(\id{S'} = \proc{Push} (\id{x}, \id{S}) \Leftrightarrow
  \proc{Pop}(\id{S'}) = (\id{x}, \id{S})\).} (assuming \(\id{S} \neq
\proc{Empty}\)). As explained previously, we need a stack whose
elements are a triple whose first component is an element, the second
the local attributes and the third the synthesised attributes in order
to model the rightmost branch of an \XML tree. For example,
\(\proc{Push} ((d_1, \varnothing, \varnothing),\proc{Empty})\) denotes
the stack after reading \(d_1\) from the stream (i.e. the first line
of the table in figure~\ref{stack_after_c2}).


\subsection{Insertion}

Let us assume a globally accessible query \(q = \query{\kappa_1}{\kappa_2,
  \dots, \kappa_{{\cal A}(q)}}\). Let \(\proc{Insert} (\varepsilon,
\id{S})\) be a pair whose first component is a (possibly empty) series
of matches and the second component is the stack resulting from the
insertion of element \(\varepsilon\) into the stack \id{S}. The
(complete) matches are obtained by reading element \(\varepsilon\)
from the input stream and combining it with previous partial matches
(while
\begin{wrapfigure}[9]{r}{0pt}
\centering
\(
\begin{array}{ccc}
\toprule
& \multicolumn{2}{c}{\text{Attributes}}\\
\cmidrule(l){2-3}
\text{Element} & \text{Local} & \text{Synthesised}\\
\toprule
d_1 & \varnothing & \varnothing\\
a_1 & \varnothing & a_1c_1, a_1b_1\\
c_2 & \mathbf{a_1b_1c_2}, a_1c_2 & \varnothing\\
\bottomrule
\end{array}
\)
\caption{Stack after adding \(c_2\).\label{stack_after_c2}}
\end{wrapfigure}
inserting it into the stack \id{S}). They are streamed out by
increasing roots and, if matching with order, by increasing
descendants, else the order is undefined (for matches with same
root). For the sake of brevity, we do not give here the loop that
implements this process of calling iteratively
\proc{Insert}\footnote{We use the algorithmic language defined by
  Cormen et al. in their famous textbook \cite{Cormen:2001}. Moreover,
  we write with a static single-assignment style for the sake of
  clarity.}, and instead focus directly on \proc{Insert} itself. For
example, consider figure~\ref{stack_after_c2}. If \id{S} is the stack
before adding \(c_2\) and \(({\cal M},\id{S}') = \proc{Insert}(c_2,
\id{S})\), then \({\cal M} = \{[1 \mapsto a_1, 2 \mapsto b_1, 3
  \mapsto c_2]\}\) and \(\id{S}'\) is the stack shown in the figure,
except that, for the sake of clarity we will continue to show the
matches in the stack in bold \emph{even if they are actually streamed
  out}. For example, in figures~\ref{stack_before_b2}
and~\ref{stack_after_b2}, the matches \(\mathbf{a_1b_1c_2}\) and
\(\mathbf{a_1b_2c_1}\) are shown inside the stack as local attributes.
\begin{codebox}
\Procname{\(\proc{Insert} (\varepsilon, \id{S})\)}
\li	\If \(\id{S} = \proc{Empty}\) \label{insert:li:empty_stack}
\RComment \(\varepsilon\) is the first element.
\li	\Then \(\Return \, (\varnothing, \proc{Push} ((\varepsilon, \varnothing,
                        \varnothing), \id{S}))\)
          \label{insert:li:not_matches}
\RComment No match yet.
	\End
\li \(((\varepsilon_1, \lambda_1, \sigma_1), \id{S_1}) 
      \gets \proc{Pop}(\id{S})\) 
    \label{insert:li:first_pop}
\RComment Let us destructure one layer of stack \(\id{S}\). 
\li	\If \(\varepsilon \sqsubset \varepsilon_1\)
    \label{insert:li:containment_in_top}
\RComment Does the top of the stack contain \(\varepsilon\)?
\li	\Then \(({\cal M}, {\cal P}) \gets \proc{Complete} (\varepsilon, \id{S})\)
    \label{insert:li:complete_stack}
\RComment New complete and partial matches.
\li \(\Return \, ({\cal M}, \proc{Push} ((\varepsilon, {\cal P},
                                          \varnothing), 
                                         \id{S}))\)
    \label{insert:li:extension}
\RComment Matches and rightmost extension.
    \End
\li  \(((\varepsilon_2, \lambda_2, \sigma_2), \id{S_2}) 
       \gets \proc{Pop} (\id{S_1})\)
     \label{insert:li:second_pop}
\RComment Destructure one more layer of stack \(\id{S}\).
\li  \(\Return \, \proc{Insert} (\varepsilon, 
                                 \proc{Push} ((\varepsilon_2,
                                               \lambda_2,
                                               \sigma_2\!\cup\!
                                               \lambda_1 \!\cup \sigma_1),
                                              \id{S_2}))\)
     \label{insert:li:expansion}
\RComment Rightmost expansion.
\end{codebox}
\noindent There are three cases when defining \(\proc{Insert}
(\varepsilon, \id{S})\): (\emph{i}) the stack \id{S} is empty because
\(\varepsilon\) is the first element in the stream, at
lines~\ref{insert:li:empty_stack}--\ref{insert:li:not_matches};
(\emph{ii}) the top element of \id{S} contains \(\varepsilon\) (which
implies the extension of the rightmost branch), at
lines~\ref{insert:li:containment_in_top}--\ref{insert:li:extension};
\begin{wrapfigure}[7]{r}{0pt}
\centering
\(
\begin{array}{c>{\hspace*{5pt}}c>{\hspace*{5pt}}c}
\toprule
& \multicolumn{2}{c}{\text{Attributes}}\\
\cmidrule(l){2-3}
\text{Element} & \text{Local} & \text{Synthesised}\\
\toprule
\varepsilon & \varnothing & \varnothing\\
\bottomrule
\end{array}
\)
\caption{\(\varepsilon\) is the first element.}
\label{first_elem}
\end{wrapfigure}
(\emph{iii}) \(\varepsilon\) and the top of \id{S} are disjoint (which
implies a rightmost expansion), at
lines~\ref{insert:li:second_pop}--\ref{insert:li:expansion}. If \id{S}
is empty, the resulting stack only contains \(\varepsilon\) with
\(\varepsilon\) as local attribute and no synthesised
attributes. There cannot be any matches, even partial. See
figure~\ref{first_elem}. Otherwise, we pop up element
\(\varepsilon_1\) with its local and synthesised attributes,
\(\lambda_1\) and \(\sigma_1\), and we get the remaining stack
\id{S_1}, at line~\ref{insert:li:first_pop}. If the top of the stack,
\(\varepsilon_1\), contains \(\varepsilon\)
(line~\ref{insert:li:containment_in_top}), then it means that we have
to extend the rightmost branch, which is achieved by pushing
\(\varepsilon\) on \id{S} (line~\ref{insert:li:extension}). The local
attributes \(\cal P\) of \(\varepsilon\) are computed by completing
the partial matches in \id{S} and complete matches can also be found
in the process, at line~\ref{insert:li:complete_stack}.
\(\varepsilon\) has no synthesised attributes since \(\varepsilon\)
has no contained elements yet. See the table in
figure~\ref{rightmost_extension_table}. Otherwise, the top of the
stack does not contain \(\varepsilon\), which means that we have to
perform a rightmost expan-

\begin{wrapfigure}[8]{R}{0pt}
\(
\begin{array}{ccc}
\toprule
& \multicolumn{2}{c}{\text{Attributes}}\\
\cmidrule(l){2-3}
\text{Element} & \text{Local} & \text{Synthesised}\\
\toprule
\multicolumn{3}{c}{\framebox[55mm][c]{\id{S}}}\\
\varepsilon & \proc{Complete} (\varepsilon, \id{S}) & \varnothing\\
\bottomrule
\end{array}
\)
\caption{Rightmost extension.}
\label{rightmost_extension_table}
\end{wrapfigure}
\noindent sion. This implies that the stack \id{S_1} must be
non\hyp{}empty; in other words, \id{S} must contain at least two
elements. Indeed, a rightmost expansion means that we grow another
branch rooted on the rightmost bran\-ch, but not at its end (else it
would be an extension), so the branch contains at least two nodes. Let
us call \(\varepsilon_2\) the top element of \id{S_1}, and
\(\lambda_2\) and \(\sigma_2\) its local and synthesised attributes;
\id{S_2} is the remaining stack, at
line~\ref{insert:li:second_pop}. Please consider the table at
figure~\ref{rightmost_expansion_0}.

\begin{wrapfigure}[9]{r}{0pt}
\(
\begin{array}{c>{\qquad}cc}
\toprule
& \multicolumn{2}{c}{\text{Attributes}}\\
\cmidrule(l){2-3}
\text{Element} & \text{Local} & \text{Synthesised}\\
\toprule
\multicolumn{3}{c}{\framebox[55mm][c]{\id{S_2}}}\\
\varepsilon_2 & \lambda_2 & \sigma_2\\
\varepsilon_1 & \lambda_1 & \sigma_1\\
\bottomrule
\end{array}
\)
\caption{Stack before expansion.\label{rightmost_expansion_0}}
\end{wrapfigure}
\noindent The rightmost expansion is realised by means of an extension
at the node where the new rightmost branch is rooted, i.e., when the
biggest element containing the new element \(\varepsilon\) is on the
top of the stack. Then the first step of the rightmost expansion
consists in inserting recursively \(\varepsilon\) in a stack equal to
\id{S} \emph{without its top element} \(\varepsilon_1\), until the
condition for an extension is satisfied. The attributes of
\(\varepsilon_1\) are not lost: they are added (by a set union) to the
synthesised attributes of the new top element \(\varepsilon_2\) (see
line~\ref{insert:li:expansion}). In terms of the \XML

\begin{wrapfigure}[8]{r}{0pt}
\(
\begin{array}{c>{\qquad}cc}
\toprule
& \multicolumn{2}{c}{\text{Attributes}}\\
\cmidrule(l){2-3}
\text{Element} & \text{Local} & \text{Synthesised}\\
\toprule
\multicolumn{3}{c}{\framebox[55mm][c]{\id{S_2}}}\\
\varepsilon_2 & \lambda_2 & \sigma_2 \cup \lambda_1 \cup \sigma_1\\
\bottomrule
\end{array}
\)
\caption{Recursive expansion.\label{rightmost_expansion_1}}
\end{wrapfigure}
\noindent tree, this amounts to move up the attributes of the
rightmost leaf and cut this leaf, and so on until an extension is
possible. Check figure~\ref{rightmost_expansion_1}. As an example,
consider the insertion of \(b_2\) (see last tree in
figure~\ref{rightmost_branches}). The stack before the insertion is
shown in figure~\ref{stack_before_b2}. Figure~\ref{stack_after_b2}
shows the result of the rightmost expansion caused by the insertion of
\(b_2\). Note that we assume in this figure that we search also for
unordered matches of \(\query{a}{b,c}\) thus \([1 \mapsto a_1, 2
  \mapsto b_2, 3 \mapsto c_1]\), i.e. \(\mathbf{a_1b_2c_1}\), is valid
in spite of \(c_1 < b_2\).
\begin{figure}[b]\centering
\subfloat[Stack before adding \(b_2\).\label{stack_before_b2}]
         {\(
\begin{array}{ccc}
\toprule
& \multicolumn{2}{c}{\text{Attributes}}\\
\cmidrule(l){2-3}
\text{Element} & \text{Local} & \text{Synthesised}\\
\toprule
d_1 & \varnothing & \varnothing\\
a_1 & \varnothing & a_1c_1,a_1b_1 \\
c_2 & \mathbf{a_1b_1c_2},a_1c_2 & \varnothing\\
a_2 & \varnothing & \varnothing\\
\bottomrule
\end{array}
\)}
\;
\subfloat[Stack after adding \(b_2\).\label{stack_after_b2}]
         {\(
\begin{array}{ccc}
\toprule
& \multicolumn{2}{c}{\text{Attributes}}\\
\cmidrule(l){2-3}
\text{Element} & \text{Local} & \text{Synthesised}\\
\toprule
d_1 & \varnothing & \varnothing\\
a_1 & \varnothing & a_1c_1,a_1b_1 \\
c_2 & \mathbf{a_1b_1c_2},a_1c_2 & \varnothing\\
b_2 & \mathbf{a_1b_2c_1},a_1b_2 & \varnothing\\
\bottomrule
\end{array}
\)}
\caption{The stack before and after inserting element \(b_2\).}
\end{figure}


\subsection{Completion}

In order to understand how the local attributes are computed (in case
of a rightmost extension), we must now define precisely function
\proc{Complete}.
\begin{codebox}
\Procname{\(\proc{Complete} (\varepsilon, \id{S})\)}
\li	\If \(\id{S} = \proc{Empty}\) \label{complete:li:empty_stack}
\RComment If the stack is empty
\li	\Then \Return \((\varnothing, \varnothing)\) 
\label{complete:li:no_matches}
\RComment then there is no completion.
	\End
\li	\(((\varepsilon_1, \lambda_1, \sigma_1), \id{S_1})
          \gets \proc{Pop} (\id{S})\)
    \label{complete:li:pop}
\RComment Destructure one layer of the stack
\zi
\RComment and combine \(\varepsilon\) with the synthesised attributes
of the top:
\li \If \(T(\varepsilon_1) = \kappa_1\)
\RComment if the top matches the query root,
\label{complete:li:top_matches_root}
\li \Then \(({\cal M}_1, {\cal P}_1) 
             \gets \proc{Combine} (\varepsilon, \sigma_1 \cup \{[1
             \mapsto \varepsilon_1]\})\)
\RComment it may combine with \(\varepsilon\).
\label{complete:li:add_combination}
\li \Else \(({\cal M}_1, {\cal P}_1) 
             \gets \proc{Combine} (\varepsilon, \sigma_1)\)
    \label{complete:li:combine}
    \End
\li \(({\cal M}_2, {\cal P}_2) 
       \gets \proc{Complete} (\varepsilon, \id{S_1})\)
    \label{complete:li:recursion}
\RComment Complete the remaining partial matches.
\li \Return \(({\cal M}_1 \cup {\cal M}_2, {\cal P}_1 \cup {\cal P}_2)\)
    \label{complete:li:union}
\RComment New partial and complete matches.
\end{codebox}
\noindent If stack \id{S} is empty, then let us return no new matches
(lines~\ref{complete:li:empty_stack}--\ref{complete:li:no_matches}). Otherwise,
let us pop element \(\varepsilon_1\), whose synthesised attributes are
\(\sigma_1\), and let \id{S_1} be the remaining stack, at
line~\ref{complete:li:pop}. The next step is to try to combine
\(\varepsilon\) with the partial matches in \(\sigma_1\) and get new
matches \({\cal M}_1\) and new partial matches \({\cal P}_1\)
(line~\ref{complete:li:combine}) but we must not forget a possible new
partial match involving only \(\varepsilon\) and
\(\varepsilon_1\). Thus we first check whether the top of the stack,
\(\varepsilon\), matches the query root, at
line~\ref{complete:li:top_matches_root}. If so, we also must try to
combine \(\varepsilon_1\) and \(\varepsilon\), at
line~\ref{complete:li:add_combination}. For example, consider the
partial match \(a_1c_2\) in figure~\ref{stack_after_c2}. Next, we
complete the remaining stack \id{S_1} with \(\varepsilon\) and get new
matches \({\cal M}_2\) and new partial matches \({\cal P}_2\)
(line~\ref{complete:li:recursion}). We merge these new matches and we
merge separately the partial matches (line~\ref{complete:li:union}).


\subsection{Combination}

Let us define now \proc{Combine}, which is the function that tries to
complete a set of partial matches with a given element. Since we deal
here with partial matches, we must be more precise here. Let us note
\(\cal D\) the function that returns the kind indexes of a given match
(the \emph{domain} of the match). For example, if \(q =
\query{a}{b,c}\) and \(\mu_q = [1 \mapsto a_1, 2 \mapsto b_1, 3
  \mapsto c_2]\), then \({\cal D}(\mu_q) = \{1,2,3\}\). Let us note
\(\overline{\cal D}(\mu_q)\) the complementary set \(\{1, \dots, {\cal
  A}(q)\} \setminus {\cal D}(\mu_q)\). If \(\mu_q = [1 \mapsto a_1, 3
  \mapsto c_2]\), then \({\cal D}(\mu_q) = \{1,3\}\) and
\(\overline{\cal D}(\mu_q) = \{2\}\). Then the match \(\mu_q\) is
complete if and only if \(\overline{\cal D}(\mu_q) =
\varnothing\). Let us note \(\mu_q \oplus i \mapsto \varepsilon\) the
extension of a match \(\mu_q\) with the binding \(i \mapsto
\varepsilon\). The operator \(\oplus\) can be formally defined, for
all \(j \in {\cal D}(\mu_q) \cup \{i\}\), as
\[
(\mu_q \oplus i \mapsto \varepsilon)(j) \triangleq
\left\{
\begin{aligned}
\varepsilon && \text{if} \, i = j\\
\mu_q(j)    && \text{otherwise}
\end{aligned}
\right.
\]
Moreover, let us assume we have a function \proc{Choose} that takes a
set of matches and returns one of them paired with the complementary
matches. (This is a way to defer to the implementation the actual
iteration order.) The following definition of \proc{Combine} returns
both ordered and unordered matches. For ordered matches only, see
further.
\begin{codebox}
\Procname{\(\proc{Combine} (\varepsilon, \sigma)\)}
\li	\If \(\sigma = \varnothing\)
\label{combine:li:empty}
\RComment If there are no partial matches
\li	\Then \Return \((\varnothing, \varnothing)\)
\label{combine:li:nothing}
\RComment then return no new matches.
	\End
\li	\((\mu_q, \sigma') \gets \proc{Choose} (\sigma)\)
\label{combine:li:arbitrary_selection}
\RComment Pick a partial match \(\mu_q\); the remaining ones are \(\sigma'\).
\li \(({\cal M}, {\cal P}) \gets 
      \proc{Combine} (\varepsilon, \sigma')\)
\label{combine:li:recursion}
\RComment Combine \(\varepsilon\) with the remaining partial matches.
\li	\If \(\exists i \in \overline{\cal D}(\mu_q).T(\varepsilon) = \kappa_i\)
\label{combine:li:new_match}
\RComment If \(\varepsilon\) completes \(\mu_q\) at index \(i\),
i.e. \(\varepsilon\) matches \(\kappa_i\),
\li	\Then \If \(|\overline{\cal D}(\mu_q)| = 1\)
\label{combine:li:one_index_remaining}
\RComment then if \(i\) was the sole unused index
\li       \Then \Return \(({\cal M} \cup \{\mu_q \oplus i \mapsto
                          \varepsilon\}, {\cal P})\)
\label{combine:li:make_complete_match}
\RComment we make a complete match,
          \End
\li       \Return \(({\cal M}, 
                     {\cal P} \cup \{\mu_q \oplus i \mapsto
                     \varepsilon\})\)
\label{combine:li:make_partial_match}
\RComment else it is a partial match.
	\End
\li	\Return \(({\cal M}, {\cal P})\)
\label{combine:li:ignore}
\RComment If \(\varepsilon\) does not extend \(\mu_q\), ignore the
partial match \(\mu_q\).
\end{codebox}
If the set of partial matches \(\sigma\) is empty
(line~\ref{combine:li:empty}), then let us return an empty set of new
matches (line~\ref{combine:li:nothing}). Otherwise, let us arbitrarily
choose any partial match \(\mu_q\) in \(\sigma\) and name the
remaining partial matches \(\sigma'\)
(line~\ref{combine:li:arbitrary_selection}). Next, let us recursively
combine \(\varepsilon\) with the remaining partial matches
(line~\ref{combine:li:recursion}). If \(\mu_q\) can be extended by
\(\varepsilon\), that is to say, at least one of its index \(i\) is
unused and \(\varepsilon\) matches \(\kappa_i\)
(line~\ref{combine:li:new_match}), then a new partial or complete
match can be made. If index \(i\) was the last unused in \(\mu_q\)
(line~\ref{combine:li:one_index_remaining}), then \(\mu_q \oplus i
\mapsto \varepsilon\) is a complete match
(line~\ref{combine:li:make_complete_match}), otherwise it is partial
(line~\ref{combine:li:make_partial_match}). If \(\varepsilon\) can not
extend \(\mu_q\) (either because it matches no node of the query or
because the sole matching nodes in the query are already matched by
nodes of the \XML tree read so far), we just ignore it
(line~\ref{combine:li:ignore}).

%\subsection{Sharing roots}

%\subsection{Ordered matches only} 
